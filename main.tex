% A pure minimalistic LaTeX-Beamer theme for everyone to use.
% Copyright (C) 2020 Kai Norman Clasen
% Edited by Lucas Saldyt

\documentclass[aspectratio=169]{beamer}
\DeclareMathSizes{12}{30}{16}{12}
% should also look nice for the classic aspectratio
% of course, than the text has to be refitted
% \documentclass{beamer} 
\usepackage[utf8]{inputenc}
\usepackage[T1]{fontenc}
\usepackage{tikz}
\usepackage{amsmath}
\usepackage[export]{adjustbox}
\usepackage[percent]{overpic}
\usepackage{epigraph}

\DeclareMathOperator*{\argmin}{\arg\!\min}
\DeclareMathOperator*{\argmax}{\arg\!\max}

\usetheme[showmaxslides, darkmode]{pureminimalistic}

\usepackage[american]{babel}
\usepackage{csquotes}
\usepackage[style=apa, backend=biber]{biblatex}
\DeclareLanguageMapping{american}{american-UoN}

% \usepackage[english]{babel}
% \usepackage[backend=biber,style=apa]{biblatex}
% \DeclareLanguageMapping{english}{english-apa}
\addbibresource{sources.bib}

% this makes it possible to add backup slides, without counting them
\usepackage{appendixnumberbeamer}
\renewcommand{\appendixname}{\texorpdfstring{\translate{appendix}}{appendix}}

\renewcommand{\logotitle}{\includegraphics[width=.2\linewidth]{logos/asu_logo_alt.png}}
\renewcommand{\logoheader}{}
\renewcommand{\logofooter}{\includegraphics[width=.15\linewidth]{logos/asu_logo_alt.png}}
\renewcommand{\emph}[1]{{\Huge \color{pureminimalistic@text@red} #1}}
\newcommand{\white}[1]{{\color{pureminimalistic@text@white} #1}}
\newcommand{\red}[1]{{\color{pureminimalistic@text@red} #1}}

\definecolor{c1}{RGB}{30, 76, 214}
\definecolor{c2}{RGB}{161, 3, 74}
\definecolor{c3}{RGB}{255, 132, 0}
\definecolor{c4}{RGB}{255, 196, 0}
\definecolor{c5}{RGB}{3, 171, 34}
\definecolor{grey}{RGB}{130, 130, 130}

\newcommand{\ca}[1]{{\color{c1} #1}}
\newcommand{\cb}[1]{{\color{c2} #1}}
\newcommand{\cc}[1]{{\color{c3} #1}}
\newcommand{\cd}[1]{{\color{c5} #1}}
\newcommand{\ce}[1]{{\color{c4} #1}}
\newcommand{\gr}[1]{{\color{grey} #1}}

\definecolor{code}{RGB}{52, 219, 235}
\newcommand{\plnr}[1]{{\color{code}$#1$}}

\title[Lifelong Learning for Planning]{Lifelong Learning for Robot Path-Planning}
\author{Lucas Saldyt}
\institute{Arizona State University} 
\date{\today}

\begin{document}

% Hello, I'm Lucas, and today I'll be presenting  "Lifelong Learning for Robot Path Planning"
\maketitle

\begin{frame}[plain]
  % This thesis focuses on environmental adaptation in robots.
  % An example robot is NASA's Perseverance rover, which was custom-engineered for mars.
  % Because of resource constraints and a desire for reliability, this rover doesn't use machine learning directly
  \begin{figure}
  \centering
  \vspace*{-1em}
  \hspace*{-3em}
  \includegraphics[height=9.5cm,keepaspectratio]{figures/perseverance.jpg}
  \end{figure}
\end{frame}

\begin{frame}[plain]
  % Instead, the rover uses an engineered navigation algorithm, or is even controlled directly by humans. (pause)
  % As an environment, Mars offers unique challenges.
  % For example, rocks can puncture the rover's wheels and sand can trap the rover.
  % The rover relies on a combination of on-board and satellite based sensor data, which is often imperfect.
  % Also, low-latency direct human control is impossible.
  \begin{figure}
  \centering
  \vspace*{-1em}
  \hspace*{-3em}
  \includegraphics[height=9.5cm,keepaspectratio]{figures/mars_surface.jpg}
  \end{figure}
\end{frame}

\begin{frame}[plain]
  % Alternatively, consider an environment like an Amazon warehouse.
  % In contrast to mars, it is easy to navigate.
  % The floor is perfectly smooth, the environment is mapped almost perfectly.
  % Communication is fast, and it is easier integrate advanced machine learning algorithms with robots.
  % However, in an environment like this, there can be multiple robots, or even human-robot interaction.
  % Humans, unlike robots, don't broadcast their telemetry, and can be unpredictable.
  % So, robots must plan very carefully to avoid dangerous collisions.
  % (pause)
  % Considering these differences between Mars and an Amazon warehouse brings us to the main goal of this thesis.
  \begin{figure}
  \centering
  \vspace*{-1em}
  \hspace*{-3em}
  \includegraphics[height=9.5cm,keepaspectratio]{figures/warehouse.jpg}
  \end{figure}
\end{frame}

\begin{frame}{\white{Core Goals \& Concepts}}
  % Conceptually, the focus of this thesis is on life-long robot learning.
  %     Modern artificial intelligence is trying to move away from custom-engineering
  %     In the context of this thesis, this means teaching robots to be generally proficient.
  %     But most importantly, this thesis focuses on the potential to specialize robots to particular environments.
  % (pause)
  % On the right, we have an image which I think is quite beautiful.
  % This is a sketch from Darwin's notebook, presumably from when he first thought of evolution.
  % It shows a tree depicting common ancestry and specialization.
  \begin{columns}[T]
      \begin{column}{.5\linewidth}
            \vspace{-0.7em}
            \emph{Learning}
            \begin{itemize}
              \item {\Medium Flexibility: adapt to new tasks}
            \end{itemize}
            \vspace{1em}
            \emph{Generalization}
            \begin{itemize}
              \item {\Medium Learn overall task proficiency?}
            \end{itemize}
            \vspace{1em}
            \emph{Specialization}
            \begin{itemize}
              \item {\Medium Learn environment details?}
            \end{itemize}
      \end{column}
      \begin{column}{.5\linewidth}
          \begin{figure}
              \centering
              \includegraphics[height=5.5cm, keepaspectratio]{figures/i_think.png}
              \caption{Where is this image from?}
              \label{fig:i_think}
          \end{figure}
      \end{column}
  \end{columns}
\end{frame}

{
\setbeamercolor{background canvas}{bg=white}
\begin{frame}[plain]
  % This is an artist's depiction of the tree of life,
  % In particular, it shows how specialization occurred over time.
  % Bacteria have the oldest common ancestry.
  % Mammals, on the other hand, are the most recent, and fill a vastly different niche.
  % In between, there are incredibly diverse species of plants, fungi, fish, reptiles, and birds.
  % This has led to at least 8.7 million species on earth
  \begin{figure}
  \centering
  \includegraphics[width=1.0\linewidth,keepaspectratio]{figures/tree_of_life.png}
  \end{figure}
  \begin{center}
      \emph{Natural evolution}
  \end{center}
\end{frame}
}

\begin{frame}[plain]{}
  % Ideally, this thesis would achieve the same specialization, but with computer programs instead of biological species.
  % Doing this is known as Evolutionary Programming
  \centering
  \vfill
  \red{\fontsize{40}{50}\selectfont Computer Programs instead of species}
  \vfill
  \Huge Evolutionary Programming
\end{frame}

\begin{frame}[plain]
  % In this slide is a teaser figure showing an ancestry tree, where programs have evolved for special tasks.
  \begin{figure}
  \centering
  \includegraphics[width=1.0\linewidth,keepaspectratio]{figures/tree.pdf}
  \end{figure}
  \begin{center}
  \emph{Evolved Computer Programs}
  \end{center}
\end{frame}

\begin{frame}[plain]{\white{Problems Considered}}
  % In particular, the task this thesis considers is *Path Planning*
  % Path planning is the task of finding a sequence of moves between points in space.
  % For example, on the left is a 2 dimensional map of Milan, Italy, and a robot has been tasked with finding a path between the Orange dot and the Purple dot.
  % Originally, this thesis focused on graph-based planning, where possible moves are specified in advance, in the form of a discrete graph.
  % However, it has moved towards sampling-based planning, where possible moves are gathered by probing the environment.
  % The methods describes in this thesis work just as well dynamic (Mars) or multi-agent planning (Amazon), but these are left for future work. 
  % They also have strong relations to symbolic regression or neural architecture search.
  \begin{columns}[T]
      \begin{column}{.5\linewidth}
          \vspace{0.5cm}
          \includegraphics[width=1.0\linewidth, keepaspectratio]{figures/one_off.pdf}
          \vfill
      \end{column}
      \hspace{1.0em}
      \begin{column}{.5\linewidth}
          \emph{Path Planning}
          \begin{vfilleditems}
              \item {\Large Graph-based}
              \vspace{1em}
              \item {\Large Sample-based \Medium (left)}
              \vspace{1em}
              \item {\color{grey} {\Large Dynamic (i.e. Mars)}}
              \vspace{1em}
              \item {\color{grey} {\Large Multi-agent (i.e. Amazon)}}
          \end{vfilleditems}
          \vspace{1em}
          {\color{grey} {\Large Symbolic Regression}}
          \vspace{1em}
          {\color{grey} {\Large Neural Arch. Search}}
      \end{column}
  \end{columns}
\end{frame}

\begin{frame}[plain]{Graph search w/ Dijkstra \white{(New York)}}
    % This image shows the behavior of a graph-based path planner, specifically Dijkstra's algorithm, on a map of New York.
    % In this case, the streets are specified in advance, and a planning algorithm is asked to find a route from the bottom of New York to the top of New York
    % Pink indicates roads that were considered
    % Blank indicates unconsidered roads
    % Yellow indicates the optimal path
    \includegraphics[width=1.0\linewidth, keepaspectratio, trim={0cm, 2cm, 1cm, 0.5cm}, clip]{figures/ny_graph_based.png}
\end{frame}

\begin{frame}[plain]{Graph search w/ A* \white{(New York)}}
    % However, the choice of algorithm can be a difference of millions of nodes
    % This algorithm is A* with a geodesic heuristic.
    % This means the algorithm knows a lower bound on how far it is from the goal at any time.
    \includegraphics[width=1.0\linewidth, keepaspectratio, trim={0.5cm 1.5cm 0 2.5cm}, clip]{figures/ny_graph_based_geodesic.png}
\end{frame}

\begin{frame}[plain]{Graph-based Algorithms \white{(Kiev, Ukraine)}}
  % The graph-based portion of this thesis considered four relatively simple algorithms.
  % These are Breadth first search, depth first search, Dijkstra's algorithm, and A* and its variants.
  % If people are interested in more advanced algorithms, Hannah Bast of the University of Frieberg has an amazing paper on the subject.
  % When using a service like Google Maps, it is more likely that it uses one of these more advanced algorithms.
  % However, I leave these for future and ongoing work.
  \begin{columns}[T]
      \begin{column}{.4\linewidth}
          \begin{vfilleditems}
              \item {\Large Breadth-first (BFS)}
              \item {\Large Depth-first (DFS)}
              \item {\Large Dijkstra/A*}
              \vspace{1em}
              {\color{grey}
              \item {\Large Goal-directed}
              \item {\Large Separator-based}
              \item {\Large Hierarchical}
              \item {\Large Bounded Hop}
              \item {\Large Hybrid Techniques}
              \vspace{1em}
              }
              \item \red{\Large \cite{bast2016route}}
          \end{vfilleditems}
      \end{column}
      \begin{column}{.6\linewidth}
      \includegraphics[height=0.9\textheight, keepaspectratio, trim={5cm 3cm 12cm 2cm}, clip]{figures/kiev.png}
      \end{column}
  \end{columns}
\end{frame}

\begin{frame}[plain]{Sampling-based search \white{(Baldur's Gate)}}
  % Next, there are sampling-based algorithms, which work by randomly sampling points in space.
  % There are two main variants: Rapidly exploring random trees and probabilistic road-maps
  % RRTs by far have the most variants, all of which address different conditions, like goal direction, a large amount of constrains, or a dynamic environment.
  % Probabilistic road maps, on the other hand, (TODO)
  % I focus mostly on simple RRT* variants
  \begin{columns}[T]
      \begin{column}{.35\linewidth}
      \vspace{0.05\textheight}
      \includegraphics[height=0.7\textheight, keepaspectratio]{figures/baldurs_rrt_one_off.pdf}
      \end{column}
      \begin{column}{.65\linewidth}
          \begin{vfilleditems}
              \item {\Large Rapidly Exploring Random Trees \red{(RRTs)}}
              \begin{itemize}
                  \item RRT, RRT*, Informed RRT*
                  \item Multi-phase
                  \item RRT-Blossom
                  \item Dynamic RRT* (RT-RRT*, RRTX, RRT#)
              \end{itemize}
              \vspace{1em}
              \item {\Large Probabilistic Road Maps \red{(PRMs)}}
              % \item {\Large \color{grey} Potential Fields}
          \end{vfilleditems}
      \end{column}
  \end{columns}
\end{frame}

\begin{frame}[plain]{RRT* \white{(Baldur's Gate)}}
    % This figure shows the voronoi regions (in orange)
    %   and convex hull (in pink) of an RRT* run
    % The voronoi regions are the basic operating principle for RRT*
    % Essentially, when a new point is sampled, it will be connected to
    %   its nearest neighbor. Voronoi regions represent which node
    %   is nearest to an *area* of the map.
    % The convex hull shows the total area explored
    \includegraphics[width=1.0\linewidth, keepaspectratio]{figures/baldurs_1k_explorer.pdf}
\end{frame}

\begin{frame}{Maps Overview {\Medium \color{white}( \cite{sturtevant2012benchmarks})}}
    % Here is an overview of the various 2D maps considered in this thesis
    % This includes maps from video games, such as:
    %   - Baldur's Gate
    %   - Dragon Age
    %   - Starcraft
    %   - World of Warcraft
    % However, it also includes obstacle maps of large cities, such as:
    %   - Boston, Shanghai, Milan
    % Finally, it includes several mazes, and even some Amazon warehouses
    \includegraphics[width=0.95\linewidth, keepaspectratio]{figures/show_maps_overview.pdf}
\end{frame}

\begin{frame}{Maps Detail {\Medium \color{white}( \cite{sturtevant2012benchmarks})}}
    % Here, we see various maps in closer detail.
    % Many maps in this dataset have unique characteristics.
    % For example, this Shanghai map, or the bottom-left Dragon Age map are both relatively open.
    % In contrast, the Boston map and Starcraft: Frozen Sea map are cluttered.
    % The Baldur's gate map, the maze, and the Dragon Age mansion map all have narrow corridors.
    % Because each map has it's own characteristics, the goal of this thesis is to specialize algorithms to each of them.
    \includegraphics[width=1.0\linewidth, keepaspectratio]{figures/show_maps.pdf}
\end{frame}

\begin{frame}{Specialization Results}
  \begin{columns}[T]
      \begin{column}{.6\linewidth}
      \centering
      \includegraphics[height=0.83\textheight]{figures/learned2.pdf}
      \end{column}
      \begin{column}{.4\linewidth}
      \epigraph{
      There is no “silver bullet” algorithm for solving all path planning problems. Heuristics that lead to massive speed-up in one scenario might be detrimental in others. Also, algorithmic parameters are mostly ad-hoc and correctly tuning them to a \red{specific environment} might drastically increase performance.
      }{\textit{
      Nikolaus Correll
      }}
      \end{column}
  \end{columns}
\end{frame}

\begin{frame}{Specialization Results}
    % Here are the results of specializing algorithms to particular maps.
    % This experiment looked at five different scenarios:
    %   Milan, Italy, (in the top left) which is a relatively cluttered map
    %   A blank map   (in the bottom right) which is intentionally open and easy to specialize to.
    % Three starcraft maps: Enigma, Turbo, and Entanglement.
    % Milan and the blank map are meant to be controls,
    %   while the starcraft maps are the main experiment.
    
    % Each experiment evolves a population of computer programs which are evaluated on several problems on a single map.
    % In the methods section, I'll explain how this evolution happens in-depth.
    
    \includegraphics[width=1.0\linewidth, keepaspectratio]{figures/learned.pdf}
\end{frame}

\begin{frame}{Milan}
    % On the Milan map, the best planner learned *NOT* to bias towards the goal,
    % And instead learned to take very large steps.
    % The worst Milan planner was one which takes very small steps, but eventually covers the search space very evenly.
    % Next, there is the best planner on the Enigma map, in the bottom left. 
    % This planner biases towards the goal 7/8 of the time, and considers 74.87% fewer nodes than the baseline planner.
    
    \includegraphics[width=1.0\linewidth, keepaspectratio]{figures/learned_split_0.pdf}
\end{frame}

\begin{frame}{Enigma/Entanglement Convergent Evolution}
    % MOST significantly, this planner is exactly the same as planner c5b4f, which
    %   evolved as the best-in-class planner on the Entanglement map. 
    % This is an example of convergent evolution.
    % Finding situations like this is the core goal of this thesis.
    % (pause)
    \includegraphics[width=1.0\linewidth, keepaspectratio]{figures/learned_split_1.pdf}
\end{frame}

\begin{frame}{Enigma Validation}
    % Here is a quick validation run showing how a single attempt differed from the initial baseline.
    \includegraphics[width=1.0\linewidth, keepaspectratio]{figures/learned_split_2.pdf}
\end{frame}

\begin{frame}{Turbo and No-Obstacles}
    % Alternatively, the Turbo planner learns to bias towards the goal roughly half the time, presumably because the space is so open.
    % Finally, the blank map planner learns to bias 89% of the time, because the open map allows it to do so.
    % These results were the first preliminary indication that goal-specialization was succeeding.
    \includegraphics[width=1.0\linewidth, keepaspectratio]{figures/learned_split_3.pdf}
\end{frame}

\begin{frame}{Specialization Results}
% This table summarizes the first specialization results across five different maps.
% Each planner reduces the number of nodes that it considers relative to the baseline.
% However, almost every planner does so in a different way.
% The Milan planner modified step size, it was also the hardest to find in terms of Edit Depth
% Based on an edit depth of 4, it is the hardest planner to find.
% The Enigma and Entanglement planners modify the bias parameter to 7/8, which controls how likely a planner is to move directly to the goal.
% Interestingly, they were also *found* in identical ways.

% Finally, the Turbo and No Obstacles planners also change the bias parameter, but to different values for each map.
\begin{table}[h]
{\small 
\caption{Best-in-class RRT* Statistics (Nodes)}
\label{environments}
\begin{center}
\begin{tabular}{l|lllll}
\hline \hline
\multicolumn{1}{c}{\bf Map}  & 
\multicolumn{1}{c}{\bf $\displaystyle -\Delta$\% Nodes} &  \multicolumn{1}{c}{\bf Edit Depth} &  \multicolumn{1}{c}{\bf Hash} & 
\multicolumn{1}{c}{\bf $r_0$ (step size)} &  \multicolumn{1}{c}{\bf $\alpha$ (bias)} &
% Name                 | Nodes   | Depth | Name | r | a | Path | Depth | Name | r | a
Milan        & \cc{$82.35$\%} & $4$ & \plnr{94e7b}  & \cb{$50,000$} & \gr{$7/4$ (>1)} \\
Enigma       & \cc{$74.87$\%} & $3$ & \plnr{49980}  & \cb{$100$}    & \cb{$7/8$} \\
Entanglement & \cc{$85.16$\%} & $3$ & \plnr{c5b4f}  & \cb{$100$}    & \cb{$7/8$} \\
Turbo        & \cc{$64.29$\%} & $3$ & \plnr{6e799}  & \gr{$50$}     & \cb{$4/9$} \\
No Obstacles & \cc{$89.47$\%} & $1$ & \plnr{70458}  & \gr{$50$} & \cb{$1/9$} \\
\hline 
\multicolumn{1}{c}{\bf Baseline} & \gr{$0.0$\%} & $0$ & {\color{code}Initial}  & \gr{$50$}     & \gr{$1.0$} \\
\hline \hline
\end{tabular}
\end{center}}
\end{table}
\end{frame}

\begin{frame}[plain]
  % So, for instance, how did planner 49980 evolve?
  % This figure shows the ancestry tree for the Starcraft Enigma experiment.
  \begin{figure}
  \centering
  \includegraphics[width=1.0\linewidth,keepaspectratio]{figures/tree.pdf}
  \end{figure}
  \begin{center}
  \emph{Evolved Computer Programs}
  \end{center}
\end{frame}

\begin{frame}[plain]{}
  % So, those were the main results of this thesis.
  % Now, I will move on to discussing how they were reached.
  % Shown here is the mandelbrot set.
  \begin{figure}
  \vspace*{-4em}
  \hspace*{-4em}
  \includegraphics[width=1.2\linewidth,keepaspectratio]{figures/mandelbrot_2.jpg}
  \end{figure}
\end{frame}

\begin{frame}[plain]{\white{Program Synthesis}}
  % A core goal of this thesis is Program Synthesis, which has a long history.
  \begin{columns}[T]
      \begin{column}{.5\linewidth}
          \vspace{1em}
          \emph{Kolmogorov}
          \newline
          \vspace{1em}
          \item \emph{Lenat}
          \newline
          \vspace{1em}
          \item \emph{Schmidhuber}
          \newline
          \vspace{1em}
          \item \emph{\cite{lake2015human}}
          \newline
      \end{column}
      \begin{column}{.5\linewidth}
          \begin{figure}
              \vspace{-3.5em}
              \hspace*{1.1em}
              \rotatebox[origin=c]{-90}{\includegraphics[height=1.0\linewidth,keepaspectratio]{figures/mandelbrot.jpg}}
          \end{figure}
      \end{column}
  \end{columns}
\end{frame}

\begin{frame}{\white{Program Synthesis \& Learning}}
    \emph{Problem} 
    \begin{itemize}
        \item What base problem is being solved?
    \end{itemize}
    \vspace{1em}
    \emph{Representation} 
    \begin{itemize}
        \item How are programs represented?
    \end{itemize}
    \vspace{1em}
    \emph{Optimization} 
    \begin{itemize}
        \item How are programs optimized?
    \end{itemize}
\end{frame}

\begin{frame}{AutoML-Zero: Evolving Machine Learning Algorithms From Scratch \white{\cite{real2020automl}}}
    \begin{vfilleditems}
        \item \Huge Learns ML practices on CIFAR-10 
        \vspace{0.7em}
        \item \Huge Tensor register machine
        \vspace{0.7em}
        \item \emph{Regularized Evolution} 
    \end{vfilleditems}
\end{frame}

\begin{frame}[plain]{AutoML-Zero \white{\cite{real2020automl}}}
\begin{figure}
\centering
\includegraphics[scale=0.37]{figures/automl_zero_main_fig.png}
\end{figure}
\end{frame}

\begin{frame}{{\color{pureminimalistic@text@white} Multiple Objectives}}
  \begin{columns}[T]
      \begin{column}{.4\linewidth}
          \begin{vfilleditems}
              \item \emph{Nodes}
              \vspace{1em}
              \item \emph{Path Length}
              \vspace{1em}
              \item \emph{Exploration}
              \vspace*{2em}
              {\centering \Huge \ldots }
          \end{vfilleditems}
      \end{column}
      \begin{column}{.6\linewidth}
      \includegraphics[width=0.9\linewidth, keepaspectratio]{figures/total_overview.pdf}
      \end{column}
  \end{columns}
\end{frame}

\begin{frame}{{Pareto Fronts}}
  \begin{columns}[T]
      \begin{column}{.4\linewidth}
          \begin{vfilleditems}
              \item {\Large Vectorized Fitness}
              \vspace{1em}
              \newline {\Large $v$ = $\{60, 155\}$}
          \end{vfilleditems}
      \end{column}
      \begin{column}{.6\linewidth}
      \includegraphics[width=0.9\linewidth, keepaspectratio]{figures/total_pareto_overview.pdf}
      \end{column}
  \end{columns}
\end{frame}

\begin{frame}{}
  \begin{columns}[T]
      \begin{column}{.5\linewidth}
        \emph{Pareto Ev.} {\color{grey} \Medium (NSGA2)}
        \begin{itemize}
            \item Multi-objective
            \item Bandwidth: {\Large $\cc{n}$}
        \end{itemize}
        \vspace{1em}
        \begin{tabular}{ll}
        &{\Large $O(\ca{g} * \cb{m} * \cc{n}^\red{2})$}\\
        &{\Large $O(\cb{m} * \cc{n})$} {\color{grey} per change}\\
        &{\Large $O(\cb{m} * \cd{p})$} {\color{grey} if sampled}\\
        \end{tabular}
      \end{column}
      \begin{column}{.5\linewidth}
        \emph{Regularized Ev.}
        \begin{itemize}
            \item Single-objective
            \item Bandwidth: {\Large $\cc{1}$}
        \end{itemize}
        \vspace{1em}
        \begin{tabular}{ll}
        &{\Large $O(\ca{g} * \cd{p})$}\\
        &{\Large $O(\cd{p})$} {\color{grey} per change}\\
        &{\Large $O(\cd{p})$} \\
        \end{tabular}
      \end{column}
  \end{columns}
  \begin{center}
  \Large
      \begin{tabular}{ll}
        Where & \ca{$g = $ generations} \\
              & \cb{$m = $ objectives}  \\
              & \cc{$n = $ total population size}\\
              & \cd{$p = $ sample size}\\
     \end{tabular}
  \end{center}
\end{frame}

\begin{frame}{}
  \begin{columns}[T]
      \begin{column}{.5\linewidth}
        \emph{Pareto Ev.}
      \end{column}
      \begin{column}{.5\linewidth}
        \emph{Regularized Ev.}
      \end{column}
  \end{columns}
\end{frame}

\begin{frame}{Pareto Evolution}
    \includegraphics[width=1.0\linewidth, keepaspectratio]{figures/paretoev.pdf}
\end{frame}

\begin{frame}{Population Dynamics: Starcraft Engima}
    \includegraphics[width=1.0\linewidth, keepaspectratio]{figures/early_pheno.pdf}
\end{frame}

\begin{frame}{Population Dynamics: Starcraft Engima}
    \includegraphics[width=1.0\linewidth, keepaspectratio]{figures/pheno.pdf}
\end{frame}

\begin{frame}{Population Dynamics: Baldur's Gate}
    \includegraphics[width=1.0\linewidth, keepaspectratio]{figures/baldurs_pheno_60.pdf}
\end{frame}

\begin{frame}{Efficient Exploration}
    \centering
    \includegraphics[width=0.85\linewidth, keepaspectratio]{figures/efficient_overview.pdf}
\end{frame}

\begin{frame}{Challenges Faced}
    \begin{vfilleditems}
    \item \Huge Test
    \end{vfilleditems}
\end{frame}

\begin{frame}{\white{Experiment Types}}
    \begin{vfilleditems}
    \item \emph{Improve}
    \item \emph{Select}
    \item \emph{Fix}
    \end{vfilleditems}
\end{frame}

\begin{frame}{Gritty Engineering Work}
    \begin{vfilleditems}
    \item \Huge Test
    \end{vfilleditems}
\end{frame}

\begin{frame}{Future Work}
    \begin{vfilleditems}
    \item \Huge Test
    \end{vfilleditems}
\end{frame}

\begin{frame}{Strong Ending}
    \centering
    \vfill
    {\fontsize{40}{50}\selectfont Final Figure and Remarks}
    \vfill
\end{frame}

\appendix % do not count the following slides for the total number
\section*{Backup Slides}

\begin{frame}[plain, noframenumbering]
  \centering
  \vfill
  {\fontsize{40}{50}\selectfont Questions?}
  \vfill
\end{frame}


\begin{frame}[plain, noframenumbering]
  \centering
  \printbibliography
\end{frame}

\begin{frame}{Population Dynamics: Baldur's Gate}
    \includegraphics[width=1.0\linewidth, keepaspectratio]{figures/baldurs_pheno_50.pdf}
\end{frame}

\begin{frame}{Milan overview}
    \centering
    \includegraphics[width=0.5\linewidth, keepaspectratio]{figures/milan_total_pareto_overview.pdf}
\end{frame}

\begin{frame}[plain, noframenumbering]{Test}
\begin{overpic}[width=1.0\textwidth,grid,tics=10]{figures/perseverance.jpg}
 \put (20,85) {\color{black}\huge$\displaystyle\gamma$}
\end{overpic}
\end{frame}

\begin{frame}[plain]{Thesis}
    \begin{vfilleditems}
      \item {\Huge Evolution is powerful}
        \begin{itemize}
          \item {\Medium Responsible for Earth's diverse, adaptive life}
          \item {\Medium \color{pureminimalistic@text@red} Evolved intelligence}
        \end{itemize}
      \item {\Huge Learned representation matters}
    \end{vfilleditems}
\end{frame}

\begin{frame}{Research Questions}
  \begin{vfilleditems}
    \item {\Huge Is Genetic Programming Competitive in 2021? {\color{pureminimalistic@text@red} Yes.}}
    % TODO: Answer each at high level
    {\color{grey}
    \item {\Huge Is crossover necessary?}
    \item {\Huge What is the role of latent code?}
    \item {\Huge How should code be mutated?}
    \item {\Huge How hard is a problem?}
    }
  \end{vfilleditems}
\end{frame}

\begin{frame}{Research Questions}
  \begin{vfilleditems}
    \item {\Huge Role of program representation?}
    \item {\Huge Role of scale and compute efficiency?}
    \item {\Huge How should code be mutated?}
    \item {\Huge How hard is a problem?}
  \end{vfilleditems}
\end{frame}

\end{document}